\documentclass[a4paper, foldmark, 12pt]{leaflet}
%\documentclass[a4paper, foldmark, nocombine]{leaflet}
\usepackage[utf8]{inputenc}
\usepackage{color}
\usepackage{hyperref}
\usepackage{times}
\usepackage{floatflt}

%% Example users
\newcommand{\UserI}{\textit{Alice}}
\newcommand{\UserII}{\textit{Ben}}
\newcommand{\UserIII}{\textit{Christine}}

%% Example movies
\newcommand{\MovieI}{\textit{The Usual Suspects}}
\newcommand{\MovieII}{\textit{American Beauty}}
\newcommand{\MovieIII}{\textit{The Godfather}}
\newcommand{\MovieIV}{\textit{Road Trip}}

\usepackage{fancyhdr}

\pagestyle{fancy}

\fancyhead{}
\fancyfoot{}

\renewcommand{\headrulewidth}{0.0pt}
\renewcommand{\footrulewidth}{0.4pt}

\fancyhead[CO,CE]{
	\vspace{-1cm}
	\begin{center}
		\includegraphics[width=9.0cm]{fig/MyMediaScreenwallBanner.jpg}
	\end{center}	
}
\fancyfoot[C]{\thepage}

%\setmargins{3cm}{2cm}{1cm}{1cm}

\headheight = 2cm



\title{MyMediaLite -- Recommender System Algorithm Library}

\author{
	\includegraphics[width=4.0cm]{fig/uni-hildesheim-400x400.jpg}\\
	Machine Learning Lab
}
\date{June 2011}

\begin{document}

\maketitle

MyMediaLite is a lightweight, multi-purpose library
of recommender system algorithms.
It addresses the two most common scenarios in collaborative filtering:
\textbf{rating prediction} (e.g. on a scale of 1 to 5 stars)
and \textbf{item prediction from implicit feedback} (e.g. from clicks or purchase actions).

\begin{center}
	\url{http://ismll.de/mymedialite}
\end{center}

\newpage

\section{MyMediaLite's Key Features}

%% TODO add support + release schedule

\begin{itemize}
	\item \textbf{Choice:}
		\begin{itemize}
			\item Dozens of different recommendation methods (see list on this flyer),
			\item methods can use collaborative, content, or relational data.
		\end{itemize}
	\item \textbf{Ready to use:}
		\begin{itemize}
			\item Includes evaluation routines for rating and item prediction;
			      quality measures MAE, NMAE, RMSE, AUC, prec@N, MAP, NDCG; and
			\item command line tools that read a simple text-based input format
                              (compatible with the one used by Apache Mahout).
		\end{itemize}
	\item \textbf{Compact:} Core library is around 100 KB ``big''.
	\item \textbf{Portable:} Written in C\#, for the .NET platform;
	      runs on every architecture where \href{www.mono-project.com}{Mono} works:
	      Linux, Windows, Mac OS X.
	\item \textbf{Free:} Free/Open Source software, distributed under the terms of the
	      GNU General Public License (GPL).
	\item \textbf{Serialization:} save and reload recommender models.
	\item \textbf{Real-time incremental updates} for most models.
\end{itemize}

\newpage

\section{Target Groups}

\subsection{Researchers}
\begin{itemize}
	\item Don't waste your time implementing methods
	      if you actually want to study
	      other aspects of recommender systems!
	\item Use the MyMediaLite recommenders as baseline methods in benchmarks.
	\item Use MyMediaLite's infrastructure as an easy
	      starting point to implement your own methods.
\end{itemize}

\subsection{Developers}
\begin{itemize}
	\item Add recommender system technologies to your software or website.
\end{itemize}

\subsection{Educators and Students}
\begin{itemize}
	\item Demonstrate/see how recommender system methods are implemented.
	\item Use MyMediaLite as a basis for you school projects.
\end{itemize}

\newpage

\section{Recommendation Tasks Addressed}

\subsection{Rating Prediction}

Popularized by systems like MovieLens, Netflix, or Jester,
this is the most-discussed collaborative filtering task in the
recommender systems literature.
Given a set of ratings, e.g. on a scale from 1 to 5,
the goal is predict unknown ratings.

\begin{center}
      \begin{tabular}{|l||c|c|c|}
        \hline
	           & \UserI & \UserII & \UserIII \\ \hline
	\hline
	\MovieI    &  5   &     & 4     \\ \hline
	\MovieII   &  3   & 4   & 3    \\ \hline
	\MovieIII  &      & \textbf{??}    & 1    \\ \hline
	\MovieIV   &  2   &     &        \\ \hline
      \end{tabular}
\end{center}


\subsection{Implicit Feedback Item Recommendation}

Getting ratings from users requires explicit actions from their side.
Much more data is available in the form of implicit feedback,
e.g. whether a user has viewed or purchased a product in an online shop.
Very often this information is positive-only,
i.e. we know users like the products they buy, but we cannot easily assume
that they do not like everything they have not (yet) bought.

%% TODO different picture
\begin{center}
	\includegraphics[width=7.0cm]{fig/interpretation_single.pdf}
\end{center}

\newpage 

\section{Implemented Methods}
\textbf{Rating Prediction}
\begin{itemize}
	\item averages: global, user, item
	\item linear baseline method by Koren and Bell
        \item frequency-weighted Slope One
	\item k-nearest neighbor (kNN):
		\begin{itemize}
			\item based on user or item similarities, with different similarity measures
			\item collaborative or attribute-/content-based
		\end{itemize}
	\item (biased) matrix factorization; factor-wise/SGD training; optimized for RMSE or MAE
\end{itemize}

\textbf{Item Prediction}
\begin{itemize}
	\item random
	\item most popular item
	\item linear content-based model optimized for BPR (BPR-Linear)
	\item support-vector machine using item attributes
	\item k-nearest neighbor (kNN):
		\begin{itemize}
			\item based on user or item similarities
			\item collaborative or attribute-/content-based
		\end{itemize}
	\item weighted regularized matrix factorization (WR-MF)
	\item matrix factorization optimized for Bayesian Personalized Ranking (BPR-MF)
\end{itemize}

\newpage

\section{Download}
Get the latest release of MyMediaLite here:
\begin{center}
	\url{http://ismll.de/mymedialite}
\end{center}

\section{Contact}
We are always happy about feedback (suggestions, bug reports, patches, etc.).
To contact us, send an e-mail to
\begin{center}
	\texttt{mymedialite@ismll.de}
\end{center}

Follow us on Twitter: {\tt @mymedialite}

\section{Acknowledgements}

\begin{floatingfigure}[r]{1.6cm}
	\vspace{-0.5cm}
	\includegraphics[width=2.1cm]{fig/uni-hildesheim-400x400.jpg}
\end{floatingfigure}
MyMediaLite was developed by Zeno Gantner,
Steffen Rendle, and Christoph Freudenthaler
at University of Hildesheim.
	
\vspace{0.4cm}

This work was partly funded by the European Commission FP7 project MyMedia
(Dynamic Personalization of Multimedia under the grant agreement no. 215006.

\vspace{0.2cm}

\begin{center}
	\includegraphics[width=4.0cm]{fig/MyMediaLogoMedium.png}
	\hspace{1.5cm}
	\includegraphics[width=2.0cm]{fig/logo-fp7.png}\\
\end{center}

\end{document}
